\documentclass[11pt, oneside]{article}
\usepackage[letterpaper, margin=2cm]{geometry}
\usepackage{Logemann}

\begin{document}
  \begin{enumerate}
    \item[\#1]
      When considering $P(A, B) = P(A | B)P(B)$, first consider if $B$ is true, this happens
      with probability $P(B)$.
      If we already know $B$, then the probability of $A$ is $P(A|B)$.
      Now both those together $A$ and $B$ give the product $P(A|B)P(B)$.

      Similarly for $P(A, B|C)$, first consider what the probability of $B$
      occurring is given $C$, that is $P(B|C)$.
      Now with both $B$ and $C$ known the probability of $A$ is $P(A|B,C)$.
      Thus the probability of both $A$ and $B$ given $C$ is $P(A|B,C)P(B|C)$.

    \item[\#2]

    \item[\#12]
      What are the differences between inference, parameter learning, and
      structure learning?
      What are you looking for in each case and what is assumed to be known?
      When might you use each of them?

      In inference you know both the structure of the Bayesian network and the
      conditional probability parameters of the network.
      You are trying to find some other probability not directly given to you
      by the network.
      You can either use exact inference and computing with the conditional
      probabilities to find the probability that you are looking for, or you
      can randomly sample the known network and estimate the probability.

      In parameter learning, you know the structure of the network but you
      do not know the conditional probabilities.
      In this case you try and use real world data to learn the paramters that
      define the conditional probabilities.
      You can do this using a Maximum likelihood approach which gives you a
      single value for each parameter, or you can use a Bayesian approach which
      gives you a confidence in you parameters as well.

      In structure learning you only have the data are trying to learning the
      conditional independence relationships of the variables.
      This can be done by picking a graph scoring function and searching the
      space of directed acyclic graphs for the best score.

      If you are given a set of data, you might first apply structure learning and
      then parameter learning to create a Bayesian network that can be associated with
      the data.
      Then it is possible to use the Bayesian network to find other previously
      unknown quantities of interest using inference.
  \end{enumerate}
\end{document}

